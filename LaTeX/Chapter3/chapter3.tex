%!TEX root = ../thesis.tex
%*******************************************************************************
%****************************** Third Chapter **********************************
%*******************************************************************************
\chapter{Literature review}
\label{chapter:litreview}
% **************************** Define Graphics Path **************************
\ifpdf
    \graphicspath{{Chapter3/Figs/Raster/}{Chapter3/Figs/PDF/}{Chapter3/Figs/}}
\else
    \graphicspath{{Chapter3/Figs/Vector/}{Chapter3/Figs/}}
\fi

\section{Prologue}

Test

\section{Energy Modelling}

Within these two approaches there exist four paradigms of models: (1) energy systems optimization models, (2) energy system simulation models, (3) power system and electricity market models and (4) qualitative and mixed methods scenarios \cite{Pfenninger2014a}. These four paradigms can be described as follows:

\begin{enumerate}
	\item These models cover the entire energy system and use optimization methods. The primary aim of these is to provide scenarios of how the system can involve.
	\item Models which cover the entire energy system using simulation techniques. These models have a primary purpose of providing forecasts of how the system may evolve.
	\item These models are focused exclusively on the electricity system. The methods and aims which underpin them vary between models. From optimization and scenarios to simulation and prediction.
	\item These models rely more on qualitative or mixed methods rather than quantitative approaches.
\end{enumerate}

In practice, it is possible for models to lie between any of these groups. Table \ref{tab:intro:modeltypes} displays the 

\begin{table}[]
	\footnotesize
	\caption{Four different model types \cite{Pfenninger2014a}}
	\label{tab:intro:modeltypes}
	\begin{tabular}{@{}lll@{}}
		\toprule
		Model family                               & Examples                                 & Primary focus                            \\ \midrule
		Energy system optimisation models          & MARKAL~\cite{Fishbone1981a}, TIMES~\cite{Giannakidis2013}, MESSAGE~\cite{Schrattenholzer1981}, OSeMOSYS~\cite{Howells2011a}        & Normative scenarios                      \\
		Energy system simulation models            & LEAP\cite{LEAP2012a}, NEMS~\cite{Grozev2005a}, PRIMES\cite{Capros2012}                       & Forecasts, predictions                   \\
		Power system and electricity market models & WASP~\cite{WASP2001}, PLEXOS~\cite{PLEXOS2013}, EMCAS \cite{Conzelmann}, ElecSim\cite{Kell}             & Operational decisions \\
		Qualitative and mixed-methods scenarios    & DECC 2050 pathways, Stabilisation wedges & Narrative scenarios                      \\ \bottomrule
	\end{tabular}
\end{table}

\section{Optimisation models}

Test

\section{Simulation models}

Test

\section{Equilibrium models}

Test

\section{Models}

Table goes here

\section{Conclusion}

Test